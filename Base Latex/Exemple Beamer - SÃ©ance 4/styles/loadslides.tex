
% Copyright (c) 2022 by Lars Spreng
% This work is licensed under the Creative Commons Attribution 4.0 International License. 
% To view a copy of this license, visit http://creativecommons.org/licenses/by/4.0/ or send a letter to Creative Commons, PO Box 1866, Mountain View, CA 94042, USA.

%~~~~~~~~~~~~~~~~~~~~~~~~~~~~~~~~~~~~~~~~~~~~~~~~~~~~~~~~~~~~~~~~~~~~~~~~~~~~~~
% Add your packages and commands to this file
%~~~~~~~~~~~~~~~~~~~~~~~~~~~~~~~~~~~~~~~~~~~~~~~~~~~~~~~~~~~~~~~~~~~~~~~~~~~~~~

%~~~~~~~~~~~~~~~~~~~~~~~~~~~~~~~~~~~~~~~~~~~~~~~~~~~~~~~~~~~~~~~~~~~~~~~~~~~~~~
\RequirePackage{palatino}
\RequirePackage[utf8]{inputenc}
\RequirePackage[T1]{fontenc}

\usefonttheme{serif}

\usepackage{styles/elegantmacros}
\usefolder{styles}
\usetheme[style=blue]{elegant}

\newcommand{\makepart}[1]{ % For convenience
\part{#1} \frame{\partpage}
}

%~~~~~~~~~~~~~~~~~~~~~~~~~~~~~~~~~~~~~~~~~~~~~~~~~~~~~~~~~~~~~~~~~~~~~~~~~~~~~~

%~~~~~~~~~~~~~~~~~~~~~~~~~~~~~~~~~~~~~~~~~~~~~~~~~~~~~~~~~~~~~~~~~~~~~~~~~~~~~~
% Figures
\RequirePackage{booktabs}
\RequirePackage{colortbl}
\RequirePackage{ragged2e}
\RequirePackage{schemabloc}

\RequirePackage{caption}
\RequirePackage{subcaption}
\RequirePackage{tabularx}
\RequirePackage{array}
\RequirePackage{multirow}


\newcolumntype{Y}{>{\centering\arraybackslash}X}
%~~~~~~~~~~~~~~~~~~~~~~~~~~~~~~~~~~~~~~~~~~~~~~~~~~~~~~~~~~~~~~~~~~~~~~~~~~~~~~

%~~~~~~~~~~~~~~~~~~~~~~~~~~~~~~~~~~~~~~~~~~~~~~~~~~~~~~~~~~~~~~~~~~~~~~~~~~~~~~
% Figures
\RequirePackage{wrapfig}
\RequirePackage{pgfplots}
\RequirePackage{graphicx}
\RequirePackage{adjustbox}
\RequirePackage{environ}
\pgfplotsset{compat=1.18}

\makeatletter
\newsavebox{\measure@tikzpicture}
\NewEnviron{scaletikzpicturetowidth}[1]{%
  \def\tikz@width{#1}%
  \def\tikzscale{1}\begin{lrbox}{\measure@tikzpicture}%
  \BODY
  \end{lrbox}%
  \pgfmathparse{#1/\wd\measure@tikzpicture}%
  \edef\tikzscale{\pgfmathresult}%
  \BODY
}
\makeatother
%~~~~~~~~~~~~~~~~~~~~~~~~~~~~~~~~~~~~~~~~~~~~~~~~~~~~~~~~~~~~~~~~~~~~~~~~~~~~~~

%~~~~~~~~~~~~~~~~~~~~~~~~~~~~~~~~~~~~~~~~~~~~~~~~~~~~~~~~~~~~~~~~~~~~~~~~~~~~~~
% Maths 
\RequirePackage{textcomp}
\RequirePackage{amsmath} 
\RequirePackage{amsthm}
\RequirePackage{mathtools}


\setbeamertemplate{theorems}[numbered] % to number


\providecommand{\H}{\mathscr{H}}      
\providecommand{\E}{\mathbb{E}}
\makeatletter
\def\munderbar#1{\underline{\sbox\tw@{$#1$}\dp\tw@\z@\box\tw@}}
\makeatother

%~~~~~~~~~~~~~~~~~~~~~~~~~~~~~~~~~~~~~~~~~~~~~~~~~~~~~~~~~~~~~~~~~~~~~~~~~~~~~~

\usepackage[T1]{fontenc}
\usepackage{amsthm}
\usepackage[french]{babel}
\usepackage{amsmath}
\usepackage{amssymb}
\usepackage{geometry}
\usepackage{graphicx}
\frenchbsetup{StandardLists=true} 
\usepackage[all]{xy}
\usepackage{dsfont} 
\usepackage{xcolor}
\usepackage{listings}
\usepackage{makeidx}
\usepackage{booktabs}
\usepackage{multirow}
\usepackage{caption}
\usepackage{subcaption}
\usepackage{appendix}
\usepackage{url}
\usepackage{environ}
\usepackage{pgfplots}

\usepackage[ruled, vlined, noend]{algorithm2e}
\SetKw{Continue}{continue}

\usepackage{tikz}
\usetikzlibrary{tikzmark}
\usepackage{tikz-3dplot}
\usepackage{tcolorbox}
\usetikzlibrary{positioning}
\newcommand{\highlight}[2]{\colorbox{#1!17}{$\displaystyle #2$}}



\def\layersep{1.5}
\def\nodesep{1}


\makeindex

\setcounter{tocdepth}{2}




\newcommand{\lemme}[1]{
	\noindent
	\begin{tabular}{|c}
		\begin{minipage}{\textwidth}
    			\raggedright{\begin{lem} #1\end{lem}}
		\end{minipage}
	\end{tabular}
}

\DeclareMathOperator*{\argmin}{arg\,min}
\DeclareMathOperator*{\argmax}{arg\,max}
\DeclareMathOperator{\Cov}{Cov}
\DeclareMathOperator{\sgn}{sgn}
\DeclareMathOperator{\tr}{tr}
\DeclareMathOperator{\MSE}{MSE}
\DeclareMathOperator{\RMSE}{RMSE}
\DeclareMathOperator{\Bias}{Bias}
\DeclareMathOperator{\TPR}{TPR}
\DeclareMathOperator{\FPR}{FPR}
\DeclareMathOperator{\Trace}{Tr}
\newcommand{\probability}[1]{\mathbb{P}\left(\displaystyle #1 \right)}
\newcommand{\variance}[1]{\mathbb{V}\left[\displaystyle #1 \right]}
%\newcommand{\expectation}[1]{\mathbb{E}\left[\displaystyle #1 \right]}
\newcommand{\expectation}[2][]{\underset{#1}{\mathbb{E}}\left[\displaystyle #2 \right]}
\newcommand{\Tr}[1]{\Trace\left(#1 \right)}
\DeclareMathOperator{\ReLU}{ReLU}
\DeclareMathOperator{\GELU}{GELU}
\DeclareMathOperator{\SiLU}{SiLU}
\DeclareMathOperator{\SwiGLU}{SwiGLU}
\DeclareMathOperator{\GEGLU}{GEGLU}
\newcommand{\independant}{\perp \!\!\! \perp}
\newcommand{\loss}[1]{\mathcal{L}\left(#1\right)}
\DeclareMathOperator{\sign}{sign}
\DeclareMathOperator{\dCE}{\mathrm{dCE}}
\DeclareMathOperator{\calibrate}{\mathrm{cal}}
\DeclareMathOperator{\smCE}{smCE}
\DeclareMathOperator{\pGap}{\mathrm{pGap}}
\DeclareMathOperator{\dual}{\mathrm{dual}}
\newcommand{\texthighlight}[1]{\textbf{\color{primary}#1}}


\definecolor{yellow}{RGB}{252, 220, 18} %FCBC12

\NewEnviron{customquote}[2]{
	\begin{center}
		\begin{minipage}{0.85\textwidth}
			\begin{textit}
				 \BODY\ \\
			\end{textit}
			\begin{footnotesize}
				--- #1 (#2)
			\end{footnotesize}
		\end{minipage}
	\end{center}
}
\newtheorem{exercice}{Exercice}
\newtheorem{proposition}{Proposition}
\newtheorem{theoreme}{Théorème}
\newtheorem{definition_self}{Définition}
\newtheorem{hypothese}{Hypothèse}
\newtheorem{corollaire}{Corollaire}



\usepackage[protrusion=true, expansion=true]{microtype} 


\tikzset{
	cuboid/.pic = {
		\tikzset{%
      			every edge quotes/.append style={midway, auto},
      			/cuboid/.cd,
      			#1
   		 }
		\draw[thick, pic actions] (0, 0, 0)  -- ++(-\cubeLayerx*\cubeScale, 0, 0) -- ++(0, -\cubeLayery*\cubeScale, 0) -- ++(\cubeLayerx*\cubeScale, 0, 0)  -- cycle;
		\draw[thick, pic actions] (0, 0, 0) -- ++(0, 0, -\cubeLayerz*\cubeScale) -- ++(0, -\cubeLayery*\cubeScale, 0) -- ++(0, 0, \cubeLayerz*\cubeScale) -- cycle;
		\draw[thick, pic actions] (0, 0, 0) -- ++(-\cubeLayerx*\cubeScale, 0, 0) -- ++(0, 0, -\cubeLayerz*\cubeScale) -- ++(\cubeLayerx*\cubeScale, 0, 0) -- cycle;
	},
	/cuboid/.search also={/tikz},
  	/cuboid/.cd,
	width/.store in=\cubeLayerx,
  	height/.store in=\cubeLayery,
  	depth/.store in=\cubeLayerz,
	scale/.store in=\cubeScale,
 	width=10,
 	height=10,
 	depth=10,
	scale=1
}


\tikzset{
	layer/.pic = {
		\tikzset{/layer/.cd, #1}
		\pgfmathsetmacro{\size}{\matrixSize / 2}
		\foreach \index in {\matrixChannels, ..., 1}{
			\pgfmathsetmacro{\scaledIndex}{(1/sqrt(\matrixChannels)) * (1 - \index)}
			\draw[thick, pic actions] (\scaledIndex, -\size, -\size) -- (\scaledIndex, -\size, \size) -- (\scaledIndex, \size, \size) -- (\scaledIndex, \size, -\size) -- cycle;
		}
		
		\pgfmathsetmacro{\index}{(1/sqrt(\matrixChannels)) * (1-\matrixChannels)}
		\draw (\index, -\size, -\size)+(-3.5pt, -3.5pt)coordinate(edge_last);
		\draw[pic actions, |-|, midway, auto] (0, -\size, -\size)+(-3.5pt, -3.5pt) -- node[sloped, below]{\matrixChannels} (edge_last);
		\draw[thick, ->] (0, 0, 0) -- (\spaceEnd, 0, 0);
	},
	/layer/.search also={/tikz},
  	/layer/.cd,
	size/.store in=\matrixSize,
	channels/.store in=\matrixChannels,
	space/.store in=\spaceEnd,
	size=3,
 	channels=1,
	space=1
}


\tikzset{
	cuboidArchi/.pic = {
		\tikzset{/cuboidArchi/.cd, #1}
		\draw[thick, pic actions] (0, -\width/2, -\height/2)  coordinate(a)-- ++(-\depth, 0, 0) coordinate(b) -- ++(0, 0, \height) coordinate(d) -- ++(\depth, 0, 0) -- cycle;
		\draw[thick, pic actions] (0, -\width/2, -\height/2) -- ++(0, \width, 0) coordinate(c)-- ++(0, 0, \height) -- ++(0, -\width, 0) -- cycle;
		\draw[thick, pic actions] (0, -\width/2, \height/2) -- ++(-\depth, 0, 0) -- ++(0, \width, 0) -- ++(\depth, 0, 0) -- cycle;
		
		\draw (b)+(-3pt, -3pt) coordinate(b1);
		\draw[pic actions, |-|, midway] (a) +(-3pt, -3pt) -- node[sloped, below]{\depthText} (b1);
		\draw (c)+(2pt, -3pt) coordinate(c1);
		\draw[pic actions, |-|, midway] (a) +(2pt, -3pt) -- node[sloped, below]{\widthText} (c1);
		\draw (d)+(-5pt, 0) coordinate(d1);
		\draw[pic actions, |-|, midway] (b) +(-5pt, 0) -- node[sloped, above]{\heightText} (d1);
	},
	/cuboidArchi/.search also={/tikz},
  	/cuboidArchi/.cd,
	width/.store in=\width,
	width text/.store in=\widthText,
  	height/.store in=\height,
	height text/.store in=\heightText,
  	depth/.store in=\depth,
	depth text/.store in=\depthText,
 	width=10,
	width text=10,
 	height=10,
	height text = 10,
 	depth=10,
	depth text = 10
}


